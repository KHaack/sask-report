(1-2 page)\\
In order to be as flexible as possible in the future, we opted for a microservice architecture. 

spring-boot netflix framework eureka server\\
Microservices should provide unified access to the external projects.\\
component diagram?\\
REST zu alle eigentlich\\
restricted to natural language\\
Commons library\\

\subsection{Repository}
As first persistent layer we use the open source project Hadoop. Hadoop is software for storing data in a cluster of distributed systems. In our case, we only use one server, but with this solution we offer an easy-to-scale system for the future. The Hadoop API did not support Java9 at the time, so we had to implement the access the Hadoop interfaces by ourselves. This was the main task of André, I supported him to configure hadoop at initial.
docker \\
Repo-ms simple interface to the Hadoop system\\

\subsection{Database}
The second persistent layer, next to Hadoop, is a Fuseki server. Apache Jena Fuseki is a SPARQL server, in our case its setup in a docker container. It uses a TDB and provides a REST interface for updating and query the database. This was the main task of Suganya and Sepide in the first semester. They implemented the database microservice as a facade. The provide a simplified access to the database and ensure that other systems are informed about changes centrally. I support the two in programming the microservice.

\subsection{Executer}
Executer, simple workflows, complex workflows\\

\subsection{Extractors}
REST zu FOX\\

\subsection{User interface}
on the UI, we rely on popular technologies such as JQuery and Bootstrap\\
JQuery plugins where possible\\
chat integration\\
data vizialization\\

\subsection{Logging}
To implement a centralized logging solution we setup the ELK stack. Consist of Elasticsearch, Logstash, and Kibana. All three are delivered in preconfigured docker containers. Elasticsearch is a search engine. Logstash is a server‑side pipeline that collects the log data from the sources and sends it to Elasticsearch. Kibana visualize the data from Elasticsearch. Each microservices writes its log into a special directory, from which Logstash retrieves the logs. This was the main task of Suganya and Sepide in the second semester and I supported them in the configuration of the docker containers and the implementation of the changes in the microservices.

\subsection{Continuous deployment}
Docker, deployment. spring-boot default docker builder\\